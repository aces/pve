\documentclass{article}

\title{PVE user's guide}

\author{Jussi Tohka \\ jussi.tohka@cs.tut.fi}

\begin{document}

\maketitle

\section{What are pve and pve3 ?}
The programs pve and pve3 are intended for the partial volume estimation
in brain MRI.  It is assumed that the pure tissue 
types are white matter (WM) , 
gray matter (GM) and CSF. The mixed tissue types are WM/GM , GM/CSF
and CSF/background. 

The program pve handles single spectral images and pve3 handles
multi-spectral data from 3 channels. The reason for distinct programs is
that many things can be done much faster if the data is single-spectral.

This document describes how to use programs, for the description of
the underlying algorithm see \cite{}. However, for your convenience,
the next section gives a brief description of the algorithm.

\section{The algorithm behind pve and pve3}
The purpose of pve and pve3 is to estimate the amount of each tissue
type present in each voxel. They do so by utilizing an indirect
algorithm based on a statistical model of the partial volume effect: 
First each voxel is given a label based on what tissue
types it is expected to contain, the labels are WM, GM, CSF, WM/GM, GM/CSF, CSF/background. Then, by using this information the
amount of each tissue type present in each voxel is estimated. For
example, if a voxel is labeled as WM in the first step we know that
it contains only white matter tissue. However, if a voxel is labeled
as mixed voxel of WM and GM we must estimate how much it contains WM
and how much it contains GM.

This being said there is one thing that must be done before the
labeling. The distributions of the pure tissue types must be
specified. This means that their parameters must be estimated. pve and 
pve3 contain two alternatives for this: 1) you can give a classified
version of the image to be processed  to the programs which will
 then estimate parameters based on the classification. 2) you can
choose EM-type algorithm for simultaneous estimation of partial volume 
fractions
and parameters. The alternative 2) is much slower!

\section{Using pve and pve3}
There are three alternatives: \\

\subsection*{Usage 1}
\begin{verbatim}
pve  -file input_file.mnc brainmask.mnc output_file_name parametersfile.pve \
[-em/-noem] [-ml/-mlonly/-noml] [-mrf beta same similar different ]
\end{verbatim}
Here, input\_file.mnc is the name of the image to be processed,
brainmask.mnc is the name of  a
volume that contains brainmask for this image (this is not optional!)
output\_file\_name is the common part of the names of output
files. There are 4 - 7 of these and what they are is explained later.
By using the switch -file you specify that parameters are given in a
file parametersfile.pve, whose format is specified in the source file
pve\_aux.c. By specifying -em you set the parameter updates on, usually 
you should use -noem (which is the default) if you give parameters in
a file. Options -ml/-mlonly/-noml specify what to output and are
explained later. By specifying the switch -mrf you can control the
Potts model used in the classification. The default values are:

\begin{center}
\begin{tabular}{c|c|c|c}
variable & type & value & notes \\
beta & double & 0.1   &  may not be negative \\
same & integer & -2   & should be smaller or equal than similar \\
similar & integer & -1 & should be smaller or equal than different \\
different & integer & 1 & should be greater or equal than similar \\
\end{tabular}
\end{center}

Note that argument parsing is primitive, and therefore you must say
\begin{verbatim}
pve -file my_image.mnc my_mask.mnc my_pve my_params.pve -noem -noml \
-mrf 0.2 -2 -1 1
\end{verbatim}
although you only want to change value of beta and you are otherwise
satisfied with the defaults.

\subsection*{Usage 2}

\begin{verbatim} 
 pve -image input_file.mnc brainmask.mnc output_filename segmented_image.mnc \ 
[-em/-noem] [-ml/-mlonly/-noml] [-mrf beta same similar different ]
\end{verbatim}
Now parameters are estimated based on the segmented image which name is
given as segmented\_image.mnc. In the one dimensional case MCD
estimator based on pruned segmented image is used as default. In this
case also, you should set -noem. 

\subsection*{Usage 3}
\begin{verbatim} 
pve -tags input_file.mnc brainmask.mnc output_filename tagfilename.tag \
[-em/-noem] [-ml/-mlonly/-noml] [-mrf beta same similar different ]
\end{verbatim}
Parameters can also be estimated based on tag points given in
tagfilename.tag. Alternatively you may use
\begin{verbatim} 
pve -tags my_image.mnc my_mask.mnc my_pve default -em
\end{verbatim}
in which case the default tagfile will be used. As you may have
noticed in this case it is advisable to set the option -em.\\


The usage of pve3 differs only in that you must specify three input
images instead of just one. And with the switch -image the parameter
estimation is done by Maximum Likelihood estimator by default. For MCD 
estimates, use Matlab instead.

\section{Output}
The next table gives descriptions of the output files. The sign *
means the output filename given in the program call. 
\begin{center}
\begin{tabular}{c|c}
filename & what? \\
*\_disc.mnc     & discrete PV classification (6 labels) \\
*\_csf.mnc      & amount of csf in each voxel based on simplified model \\
*\_gm.mnc       & amount of gm in each voxel  based on simplified model \\
*\_wm.mnc       & amount of wm in each voxel  based on simplified model \\
*\_exactcsf.mnc & amount of csf in each voxel based on complete model \\
*\_exactgm.mnc  &  amount of gm in each voxel based on complete model \\
*\_exactwm.mnc  &  amount of wm in each voxel based on complete model \\
\end{tabular}
\end{center}

When calculating partial
volume fractions we can use the complete model or the simplified
model. There is only little difference in the results, usually in the
favor of the complete model. However, the complete model is a bit 
slower to compute and might be more sensitive for errors in the parameter estimation. 

By saying -noml (which is default) in the program call, you get 
only the estimates based on the simplified model, by saying -mlonly  you get 
only the estimates based on the complete model and by saying -ml you
get the both estimates.

\end{document}